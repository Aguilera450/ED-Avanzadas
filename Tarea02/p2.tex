\textbf{Pregunta 1.}
%\noindent
Muestra que dado un conjunto $T$ de $n$ nodos $x_1, x_2, \dots, x_n$ con valores y prioridades distintas, el árbol treap asociado a $T$ es único. Hint: utiliza inducción sobre $n$.\newline

\begin{proof}
  Procedamos por inducción en el número de nodos. Observemos que para un nodo

  %%%%%%%%%%%%%%%%%%%%%%%%%%%%%%%%%%%%%%%%%%%%%%%%%%%%%%%%%%%%%%%%%%%%%%%%%% FIGURE 1
  \begin{figure}[ht!]
    \centering
    \begin{tikzpicture}
      %% Primer bloque:
      \node(A) [blueV, label=180:$q$] at (0,0){};
    \end{tikzpicture}
  \end{figure}
  
  se cumple que el árbol Treap asociado a $q$ (en este caso) es
  único.\newline

  Supongamos que para un conjunto $T' = \{q_1, \dotsm, q_k\}$ de tamaño
  $k < n$ con valores y prioridades distintas se cumple que el árbol Treap asociado
  a $T'$ es único\footnote{Hipótesis de inducción.}.\newline

  
\tikzset{
itria/.style={
  draw,dashed,shape border uses incircle,
  isosceles triangle,shape border rotate=90,yshift=-1.45cm},
rtria/.style={
  draw,dashed,shape border uses incircle,
  isosceles triangle,isosceles triangle apex angle=90,
  shape border rotate=-45,yshift=0.2cm,xshift=0.5cm},
ritria/.style={
  draw,dashed,shape border uses incircle,
  isosceles triangle,isosceles triangle apex angle=110,
  shape border rotate=-55,yshift=0.1cm},
letria/.style={
  draw,dashed,shape border uses incircle,
  isosceles triangle,isosceles triangle apex angle=110,
  shape border rotate=235,yshift=0.1cm}
}

\begin{center}
  \begin{tikzpicture}[sibling distance=5cm, level 2/.style={sibling distance =2cm}]
    \node[draw,fill=black] {}
         { node[itria] {$T'$} };
  \end{tikzpicture}
\end{center}

Ahora, observemos que un árbol Treap $T$ de $n$ nodos tiene cómo sub-hijo izquierdo un sub-árbol,
digamos $T_1$, y cómo sub-hijo derecho un sub-árbol, digamos $T_2$. A continuación se muestra un bosquejo
\begin{center}
\begin{tikzpicture}[sibling distance=5cm, level 2/.style={sibling distance =2cm}]
  \node[circle,draw,fill=blue] {}
  child{ node[draw, fill = black] {}
    { node[itria] {$T_1$} } 
  }
    child{ node[draw, fill = black] {}
    { node[itria] {$T_2$} } 
  };
\node[draw] at (3,-5) 
{
\begin{tabular}{cl}
\tikz\node[circle, fill = blue] {}; & Raíz \\
\tikz\node[draw,  fill = black] {}; & Sub-árboles
\end{tabular}
};
\end{tikzpicture}
\end{center}

Podemos notar que $T_1$ y $T_2$ tienen una cantidad de nodos menor que $n$. Por nuestra hipótesis
de inducción $T_1$ y $T_2$ es único, lo que implica que $T$ es único.
\end{proof}
%\subsection*{Respuesta}

%<Tu respuesta aquí>

%\bigskip
