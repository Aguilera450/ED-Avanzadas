\textbf{Pregunta 9.} Describe el proceso de dinamizar el árbol de rangos de dimensión 3.

$\rhd$ \textbf{Solución:} Para este ejercicio tendremos bosques que obedezcan a su codificación
binaria, esto es, tendremos árboles de distinto tamaño que contengan distintos elementos. ¿Cómo
logramos esto?

\begin{enumerate}
\item Inicialmente tomemos la representación binaria de nuestro bosque, esta representación será
en forma de cadena cómo la siguiente
\[10101 \Rightarrow 2^4 \cdot 2^3 \cdot 2^2 \cdot 2^1 \cdot 2^0\]
que significa que tenemos un árboles de rango con 1 nodo, 4 hojas, 16 hojas respectivamente. Nuestro caso base
siempre será $2^0 = 1$ nodo. A partir de aquí construiremos las operaciones \textbf{inserta}, \textbf{borra}, \textbf{colsulta}. El resto de operaciones
se aplicaran a cada árbol del bosque cómo usualmente se aplican.
\item \textbf{Inserta.} Insertaremos punto a punto. Entonces verificamos si ya existe un árbol de rangos con un nodo (en el árbol principal).
Si no existe, entonces nuestro elemento pasa hacer este nodo con referencias a sus subárboles en Y y Z. Si existe un árbol con un sólo nodo,
entonces unimos nuestro nodo a este formando un nuevo árbol de codificación $2^1$. Repetimos este procedimiento consultando que no haya más de un
árbol con una misma codificación, en el momento que lleguemos a un árbol con codificación distinta tal que no haya codificaciones repetidas terminamos.
\item \textbf{Borra.} Esta operación es un poco más sencilla, pues basta verificar si hemos eliminado $o(n)$\footnote{o pequeña. Equivaldría a verificar una fracción lineal.} elementos al menos para reconstruir
nuestro bosque.
\item \textbf{Consulta.} Tomemos en cuenta que no tenemos un sólo árbol. Tenemos un bosque de árboles. Basta consultar en cada árbol hasta
encontrar el elemento requerido o uno suficientemente cerca (en caso de no tener el elemento en ningún árbol). ¿Cuánto nos toma esto? La
relación entre la codificación binaria y el número de árboles nos garantiza que el número de elementos en el bosque (árboles) es logarítmico
respecto del número de hojas en el conjunto de árboles. Esto nos garantiza consultar del orden de $\mathcal{O}(\log^{d + 1} n)$ si no es un
árbol de rangos con cascada.
\item El resto de operaciones son las usuales aplicadas a cada elemento del bosque.
\end{enumerate}

\hfill $\lhd$
