\textbf{Pregunta 5}
Sea $S$ un conjunto de $n$ segmentos de línea sin cruces entre ellos.
Queremos responder rápidamente a consultas del tipo: dado un punto $p$
encontrar al primer segmento en $S$ por el que pasa el rayo vertical con
origen en $p$ y dirección hacia arriba. Da una estructura de datos para
resolver este problema. Acota el tiempo de consulta y el espacio requerido
por tu estructura. ¿Cuál es el tiempo de pre-procesamiento?\newline

$\rhd$ \textbf{Solución.} Para este problema utilizaremos un árbol de segmentos
modificando, dónde podamos consultar las intersecciones de la línea generada
con el resto de líneas que ya se encuentran en el árbol, entonces ¿Cómo realizamos
las consltas?\newline

1. Generamos la línea $l$ a partir de $p$, entonces consultamos las listas izquierdas y derechas
de la raíz, si la línea intersecta, al menos, un segmento entonces hemos encontrado lo deseado.\newline

2. Si la línea es menor (se encuentra a la izquierda) del nodo que estemos visitando, entonces descendemos
por el hijo izquierdo y verificamos.\newline

3. Si la línea es mayor (se encuentra a la derecha) del nodo que estemos visitando, entonces descendemos
por la derecha y verificamos.\newline

4. Si verificamos y encontramos intersecciones, entonces hemos encontrado lo deseado.\newline

5. Si descendemos hasta alguna hoja y no encontramos intersecciones, entonces no hay intersecciones y terminamos.
$\lhd$
