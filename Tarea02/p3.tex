\textbf{Pregunta 2}
Se pueden utilizar las estructuras de búsqueda de rangos ortogonales para determinar si un punto particular $(a,b)$ está en un conjunto dado, haciendo una consulta al rango $[a:a] \times [b:b]$.
\begin{enumerate}
   \item Prueba que hacer una consulta así en un árbol $KD$ toma tiempo $O(\log n)$.
   \item ¿Cuál es la complejidad para una consulta así en un árbol de rangos?
\end{enumerate}

$\rhd$ \textbf{Solución:} Para este problema dividamos la solución en dos posibles casos:
\begin{enumerate}
\item[$a$)] Sabemos que hacer una consulta, en un árbol $Kd$, para un rango en
general nos toma $\mathcal{O}(\sqrt{n} + k)$. Sin embargo, consultar un punto
en un árbol $Kd$ sería equivalente a particionar la nuve de puntos por la mitad
y preguntarnos en qué parte queda nuestro punto distinguido, llamemosle $q$. Así,
descartamos aproximadamente la mitad de puntos en dónde no se encuentra $q$. Luego,
subdividimos nuevamente el conjunto de puntos restantes en $2$ y nos preguntamos
en qué parte se encuentra $q$ y podemos descartar la parte en la que no se encuentre.
De esta manera y recursivamente nuestro espacio de búsqueda se reduce a la mitad cada
vez, esto equivale hacer un recorrido de la raíz de nuestro árbol $kd$ hacia las hojas
en búsca del punto $q$. Como cada vez descontamos la mitad del conjunto de puntos
de búsqueda, tenemos la recurrencia $T(n) = 1 + T(\frac{n}{2})$ de manera vértical y
$T(n) = 1 + 2T(\frac{n}{4})$ de manera horizontal, que sabemos que nos genera
un orden logarítmico en base 2 (equivale a bajar por el árbol). Cómo el número de
consultas es igual a $1$, entonces $k = 1 \in \mathcal{O}(1)$. Por tanto, la complejidad
de esta consulta es $\mathcal{O}(\log n)$.
\item[$b$)] En este caso, tenemos una complejidad general de $\mathcal{O}(\log^2 n + k)$,
cómo solo estamos consultando un punto y no un rango, entonces $k = 1 \in \mathcal{O}(1)$.
En la consulta debemos bajar por el árbol hasta encontrar $q.X$ y bajar su árbol ``colgado''
o su árbol asociado en $y$, hacer esto es igual a $\log m$ dónde $m$ es la altura del árbol
asociado con $m \not= n$, pues cada nivel y nodo tiene un árbol compacto de tamaño constante.
Por tanto, la complejidad esta contenida en $\mathcal{O}(\log m \cdot \log n) = \mathcal{O}(\log n)$
con $m$ constante respecto de $n$.

\end{enumerate}
\hfill $\lhd$
%\subsection*{Respuesta}
%<Tu respuesta aquí>


%\bigskip
