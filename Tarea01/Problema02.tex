\textbf{\textcolor{blue}{2.}} Tenemos dos árboles binarios de búsqueda con O(n) elementos cada uno y queremos construir una lista de tamaño O(2n)  con los elementos de ambos árboles de tal manera que la lista resultante este ordenada de mayor a menor. Diseña un algoritmo de complejidad O(n) para obtener esta lista, justifica la complejidad y explica claramente tu algoritmo con un ejemplo. Nota: Toma en cuenta que el árbol binario de búsqueda puede tener elementos repetidos. (2 puntos)\newline


$\rhd$\textbf{Solución.} El algoritmo solicitado es el siguiente: para 2 pilas $p_1$ y $p_2$,
con una lista $l$.
\begin{enumerate}
\item De nuestros dos árboles, digamos $T_1$ y $T_2$, recorramos ambos con \textit{DFS in-order}
  y guardemoslos valores devueltos en las pilas $p_1$ y $p_2$ respectivamente para cada árbol.\newline

  \textbf{Obs.} Cómo $T_1$ y $T_2$ son \textit{BST}, entonces el recorrido \textit{DFS in-order}
  nos regresa sus valores en orden ascendente desde el fondo de la pila al tope.
\item Sacamos los valores de la pila de manera paralela, siempre insertando el tope mayor de
  sus respectivas pilas a $l$. Si los topes, digamos $t_1$ y $t_2$, son iguales es indistinto e
  insertamos ambos a $l$. Si una de las dos pilas termina sus elementos, entonces vaciamos la
  restante y vamos insertando sus elementos en $l$.
\end{enumerate}
\textit{Análisis de complejidad.} A continuación se detalla el análisis de tiempo:
\begin{enumerate}
\item Recorrer una gráfica mediante DFS es del orden de $\mathcal{O}(|V| + |E|)$, en este caso
  partícular recorremos dos árboles y por tanto su complejidad es del orden de sus nodos, por lo cuál
  tenemos una complejidad contenida en $\mathcal{O}(\cdot|V_{T_1}| + \cdot|V_{T_2}|) = \mathcal{O}(2\cdot|V|) = \mathcal{O}(n)$. Nuevamente, insertar los elementos en las respectivas pilas es una operación del
  orden de $\mathcal{O}(n)$.
\item Sacar los elementos de las pilas e insertarlos a $l$ nos cuesta $\mathcal{O}(2n) = \mathcal{O}(n)$.
  Pues basta sacar los elementos de la pila e insertar en $l$ (dos operaciones por elemento).
\end{enumerate}
Así, concluimos que nuestra complejidad está contenida en
\[\mathcal{O}(n) + \mathcal{O}(n) = \mathcal{O}(n).\]
\hfill $\lhd$
